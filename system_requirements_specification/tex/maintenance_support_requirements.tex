Maintenance and support requirements define the necessary measures to ensure that the UR20 robotic palletizing system remains fully functional after deployment. Effective support includes troubleshooting guides and accessible source code for any required modifications. Additionally, maintenance teams must be equipped with appropriate tools to service the robot system.

\subsection{Maintenance Procedures for UR20 and Conveyor System}
\subsubsection{Description}
Routine maintenance must be performed on the UR20 robot and the conveyor system to ensure proper functionality. This included for wear and tear on the bellow cups, the end effector itself, the robot arm, and conveyor belt mechanism.
\subsubsection{Source}
LJCJ Team
\subsubsection{Constraints}
Constraints include the availability of spare parts and in case of gripper needing full repair the design schematics will be provided. 
\subsubsection{Standards}
ISO 9283:1998 Manipulating industrial robots — Performance criteria and related test methods for industrial robots
ISO 10218-1:2011 Robots and robotic devices — Safety requirements for industrial robots 
\subsubsection{Priority}
High

\subsection{Support Documentation and Troubleshooting Guides}
\subsubsection{Description}
Detailed troubleshooting manuals and user guides will be provided based on the team's experience when working with the UR20. This will include known issues that may arise and step-by-step instructions on how to resolve them. Digital access to the resources will be available.
\subsubsection{Source}
LJCJ Team
\subsubsection{Constraints}
Documentation must be kept up to date with any changes to the system. such as software updates or any change in hardware. 
\subsubsection{Standards}
IEEE 1063-2001 IEEE Standard for Software User Documentation - content on the software will be provided to the user and meets proper documentation practices
\subsubsection{Priority}
High

\subsection{Source Code availability }
\subsubsection{Description}
The source code for the UR20 control system will be available to maintainers for debugging and future updates. A version control system will be utilized to track changes in the software and maintain organization. 
\subsubsection{Source}
LJCJ Team
\subsubsection{Constraints}
Constraints include ensuring the version control system is maintained properly, only allowing authorized users to modify code.
\subsubsection{Standards}
ISO/IEC/IEEE International Standard - Software engineering - Software life cycle processes - Maintenance (maintain software)
\subsubsection{Priority}
High

