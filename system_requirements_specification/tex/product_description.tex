This section provides the reader with an overview of UR20. The primary operational aspects of the product, from the perspective of end users, maintainers and administrators, are defined here. The key features and functions found in the product, as well as critical user interactions and user interfaces are described in detail.

\subsection{Features \& Functions}
The UR20 will perform a palletizing application using an added vacuum gripper to the end of the arm as well as a QR code reader that will store the package information, such as where it needs to be placed. Once the data is processed, the arm will place the box according to its size. As seen in the system, the griper will be made up of 20 bellow cup suction gripers  attached to an air compressor

\subsection{External Inputs \& Outputs}
The UR20 will perform a palletizing application with an added vacuum gripper at the end of the arm. The data needed to process this palletizing application will be the QR code that will communicate via serial communication to the robot arm. This information will be processed by the robot arm's code, which will contain information such as the size and position of the box needed to be placed in the pallet. The vacuum gripper will be used to pick up the box. The UR20 will be used to move the object to the apparatus location. This QR code reader will store the package information, such as where to place it. Once the data is processed, the arm will place the box accordingly. 

\subsection{Product Interfaces}
The end user will have access to a Teach Pendant which is a tablet-like screen that can be used to do maintenance on the UR20. From this tablet, one can program the robot, run a program, or configure robot installation. This allows for easy adjustments to the UR20's behavior.