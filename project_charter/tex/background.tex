This project explores the development and implementation of a palletizing application using a UR20 collaborative robot arm, designed to mimic a full-scale assembly line operation. The system integrates a conveyor belt that transports boxes to a designated stop point. 

A photoelectric sensor (photo-eye) monitors this stop point and halts the conveyor when the sensor beam is interrupted by an incoming box. The photo-eye is connected to a relay module, which controls the start/stop function of the conveyor belt. Once the conveyor is stopped, the UR20 robot arm is triggered to pick up the box and place it onto a pallet, following a predefined grid pattern to ensure organized and stable stacking. 

To prioritize safety in human-robot interaction, a safety scanner is implemented to monitor the surrounding area. The scanner automatically stops the robot’s operation if a person enters the predefined safety zone. As an additional safety enhancement, a camera is mounted on the side of the system to monitor the scanner’s blind spot or "dead zone." This camera serves as a secondary emergency stop (E-stop) by detecting any human presence within this range and halting the robot's activity accordingly.


