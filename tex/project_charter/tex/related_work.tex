
Automation is an emerging state-of-the-art technology in manufacturing and logistics, advancing significantly year after year. These solutions are available in various forms, including academic research, commercial use, and prototypes. There are several commercial solutions to optimizing work efficiency with automated robots, each with their own strengths and weaknesses. One of the large commercially available solutions is Robotiq's robotic palletizing solution, that can be installed in days and resolve production instability \cite{Robotiq2024}. Despite the advantages, solutions such as these are very costly where the palletizing application of the RV8 would cut costs.

Robotic palletizing is also a popular topic in academic research, focusing on the algorithms for object recognition, position control, and planning \cite{Xu2022}. Additionally, power consumption plays a large role in the cost of operation and is a field that is researched to reduce operating power consumption \cite{Deng2022}. Furthermore, while many solutions claim to be easily deployable, they often require substantial training on how to operate the system. The UR20 palletizing application will tackle this problem with the process and set-up being documented and easy to use even for those with little programming knowledge. Industrial solutions exist to face growing demands to improve efficiency, utilizing industrial wireless sensor networks for rapid deployment and flexibility \cite{Gungor2009}. Lastly, a major factor in the efficiency of the palletization application is the vision of the robot. The UR20 implementation will be equipped to recognize workers or other obstacles approaching the working area in an effort to avoid workplace accidents. \cite{Abdullah2022}. Our UR20 robot will feature a simple suction-powered gripper and an easily-modifiable placement algorithm so that it can be installed and set up with minimal hassle.
