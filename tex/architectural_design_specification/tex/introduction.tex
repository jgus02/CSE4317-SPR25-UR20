This project's goal is to program a collaborative palletizing robot intended for industrial use. The main component of the robot will be the PLC (programmable logic controller) designed for use with the UR20 cobot (collaborative robot) arm, and the secondary components will be the UR20 arm itself, a safety system, a conveyor belt. Each component will be an individual layer, described further later in this document. The intended product of the project will be a robot that can identify the location and position of boxes on a conveyor belt and stack them optimally, while being safe for humans to work around. The focus is the arm, but other requirements in the intended working environment (such as a conveyor belt and pallets) will be provided for testing.

The main component, the PLC, will be the control module for the entire system. All other components report back to and are controlled by the PLC. The second most important component is the UR20 arm, which is designed to work with the PLC controller. The arm will manipulate and palletize the boxes in response to the commands given by the PLC.

The PLC will make decisions based on the safety sensor system. The safety sensor system will detect human presence and inform the PLC whether it is safe to operate, ideally slowing down the movement of the UR20 arm when a human is nearby.

The conveyor belt is an external system. It constantly feed boxes to the UR20 arm, with the intention of mimicking an industrial environment. Ideally, it will be connected to the PLC's start/stop and safety system instead of having its own external controls, but the conveyor belt is only barely inside of the scope of this project, as it is likely that anyone installing this arm would have their own conveyor belt set up. Therefore, the conveyor belt is a low priority for this project. 