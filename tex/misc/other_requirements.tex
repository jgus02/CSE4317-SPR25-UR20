The UR20 in a palletizing application will require a conveyor belt, vacuum gripper attachment, and E-stop button. These components must be configured to work with the UR20 and each other for each new environment and job (with the exception of the E-stop button, which will always perform the same task in every situation).

\subsection{E-stop button}
\subsubsection{Description}
The E-Stop will completely halt all operations of the UR20, conveyor belt, and vacuum generator. 
\subsubsection{Source}
LJCJ Team Decision
\subsubsection{Constraints}
Some components may not take kindly to having the power cut unexpectedly. This should be considered before testing of the E-stop button begins.
\subsubsection{Standards}
N/A
\subsubsection{Priority}
Critical

\subsection{Vacuum Gripper Attachment}
\subsubsection{Description}
A vacuum gripper must be constructed in order for the UR20 to fulfill its basic purpose. Once built, it shall provide appropriate suction power in order to securely transport a box, and a release of that suction in order to deposit the box.
\subsubsection{Source}
LJCJ Team Decision
\subsubsection{Constraints}
The vacuum gripper's completion date is dependent on shipping times and the success of prototypes.
\subsubsection{Standards}
N/A
\subsubsection{Priority}
Critical

\subsection{Conveyor Belt}
\subsubsection{Description}
The conveyor belt must be configured to operate at a suitable speed, one that can keep up with the UR20's max speed while not spilling over when the UR20 slows down to accommodate a human presence. It must be placed in a location that is marked, as if its position significantly changes the UR20 may have difficulty recognizing the boxes.
\subsubsection{Source}
LJCJ Team Decision
\subsubsection{Constraints}
This is dependent on the conveyor belt's adjustability. In the case that a variable speed cannot be achieved, the UR20 may be forced to operate at a low speed at all times.
\subsubsection{Standards}
N/A
\subsubsection{Priority}
Moderate